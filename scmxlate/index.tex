% last change: 2012-04-21

\input tex2page
\input plainsection

\let\TZPtexlayout 0
\let\n\noindent
\let\f\numberedfootnote

\ifx\shipout\toHTML
\let\oldsection\section
\def\section{\eject\oldsection}
\fi
\let\re\subsection


\advance\hoffset .75 true in
\advance\hsize -1.5 true in

\title{scmxlate}

%\ignorenextinputtimestamp

\centerline{\urlh{scmxlate.tar.bz2}{\ifx\shipout\totheWeb
Download \fi Version 20120421}} % last modified

\centerline{\urlh{../index.html}{Dorai Sitaram}}

\medskip

Scmxlate is a configuration tool for software
packages written in Scheme.

Scmxlate provides the package author with a strategy
for programmatically specifying the changes required to
translate the package for a variety of Scheme dialects
and Common Lisp, and a variety of operating systems.
The end-user simply loads {\em one} file into
their Scheme or Common Lisp, which triggers the entire
configuration process with little or no further
intervention.

Thus, there are two types of user for Scmxlate:

\item{\bull} The end-user of an Scmxlate-configured package, who
relies on the
Scmxlate program to perform the configuration
for their system; and
\item{\bull} the package author, who uses the Scmxlate
methodology to specify an executable form of the
configuration details for the package.

The package author is still required to know a lot more
about the configuration process than the end-user, even
with Scmxlate helping the former.
%(The Common Lisp half
%of Scmxlate uses \urlh{scm2cl.html}{Scm2cl},
%which is included in the Scmxlate distribution.)
The advantage to using Scmxlate is that the several
end-users can each configure the product to their
different systems by following the same simple
step.

Section~\ref{useconfig} describes the use of Scmxlate
to execute an already written configuration, and is all
the information you will need if you are an
end-user of packages that have Scmxlate configurations.

Sections~\ref{writeconfig} and \ref{glossary} are for
package authors, and describe the method and the
language used to write an Scmxlate configuration.

\bigbreak

\noindent{\bf Contents}

\tableofcontents

\ifx\shipout\totheWeb\else
\vfill\eject
\fi

\section{Using an Scmxlate configuration}
\label{useconfig}

Scenario: You are an end-user who has just downloaded a
Scheme package, say,
\urlh{../tex2page/tex2page-doc.html}{TeX2page}.
The package author claims to have included the Scmxlate
configuration details in the package.  What do
you do?

First, you need to have Scmxlate installed on {\em
your} system.  Get the \urlh{scmxlate.tar.gz}{Scmxlate tarball} and unpack
it, creating a directory called \p{scmxlate}.  Place
this directory in its entirety in a place that is
convenient to you.  Among the files in this directory
is the file \p{scmxlate.scm}.  Note down
its {\em full} pathname so you can refer to it from
anywhere on your filesystem.
Just to make it concrete, let's assume you put the
\p{scmxlate} directory in \p{/usr/local/lib}.  Then the
full pathname to remember is

\p{
/usr/local/lib/scmxlate/scmxlate.scm
}

Now to configure the TeX2page package.  Unpack it
and \p{cd} to its directory.

For each Scheme file {\em filename}  that is to be
translated, there may (but not necessarily) be a
user-configuration file \p{scmxlate-}{\em filename} in
the top directory.  If the instructions that came with
the package suggest you edit them, do so.  In our
example package, there is only one user-configuration
file, it is called \p{scmxlate-tex2page}, and it
doesn't seem to require any edits from the casual user.

Start your Scheme or Common Lisp in the top directory
(being in that directory is important!).  In your
Scheme (or Common Lisp), type

\q{
(load "/usr/local/lib/scmxlate/scmxlate.scm")
}

\n where the \q{load} argument is of course the correct
pathname of the file \p{scmxlate.scm} for your
setup.

Scmxlate may ask you a few questions.  A
choice of answers will be provided, so you don't need
to be too creative.  When Scmxlate finishes, you
will be left with a version of the package tailormade
for you.

\section{Writing an Scmxlate configuration}
\label{writeconfig}

%{\color{red}\relax Nothing in this section is
%stable.  The implementation and user interface is
%extremely likely to change in order to make the
%documentation easier to write.}
%
%In the following, we will assume that Scmxlate was
%unpacked in \p{/usr/local/lib}, so the full pathname of the
%\p{scmxlate.scm} file is
%\p{/usr/local/lib/scmxlate/scmxlate.scm}.

\subsection{A minimal configuration}

Let us say you have a number of Scheme files in
a directory that you intend to package as a
distribution.  For specificity let's say the name of the
directory is \p{pkgdir} and you have three Scheme files in it,
viz, \p{apple}, \p{orange.scm}, and \p{banana.ss}.
There is no restriction on the names of these Scheme
files: They may have any or no extension.  An end-user
of your distribution will unpack it to produce a
\p{pkgdir} of their own with the three Scheme files in
it.

Let us now say that you wrote the Scheme files in the
MzScheme dialect of Scheme, but that the end-user
uses the Guile dialect of Scheme.  In order for them to
be able to create Guile versions of your files, you
need to provide in \p{pkgdir} some configuration
information.  This can be done as follows:

Create a subdirectory called \p{dialects} in
\p{pkgdir}.  In the \p{dialects} subdirectory,
create a file called \p{files-to-be-ported.scm}
containing the names of the Scheme files to be
translated,  viz:

\q{
"apple"
"orange.scm"
"banana.ss"
}

\n and a file called \p{dialects-supported.scm} containing
the line

\q{
guile
}

\n The symbol \q{guile} of course stands for the Scheme
dialect Guile.

The Guile-using user can now start Guile in \p{pkgdir},
and load \p{scmxlate.scm} (using the appropriate
pathname for \p{scmxlate.scm} on their system, as
described in Section~\ref{useconfig}).  Scmxlate will learn
from \path{dialects/files-to-be-ported.scm} that the files
\p{apple}, \p{orange.scm}, and \p{banana.ss} need to be
translated.  It will ask the user what the dialect is,
offering as choices the dialects listed in
\path{dialects/dialects-supported.scm}, plus a catch-all
dialect called Other:\f{The astute reader may wonder
why Scmxlate needs to explicitly ask the user what the
target dialect is, when it is already running on it!
Unfortunately, since the Scxmlate code is necessarily
written in a style that must load in all Schemes, it
cannot portably determine the identity of the
particular Scheme dialect it is currently running on.}

\p{
What is your Scheme dialect?
     (guile other)
}

The user types \p{guile} in response.  Scmxlate now
understands that it is to create Guile translations of
the three files, and proceeds to do so.  By default,
the translation-result files are created in the
\p{pkgdir} directory and have the same names as the
original but with the prefix \p{my-} attached.  Thus,
in this case, their names are \p{my-apple},
\p{my-orange.scm}, and \p{my-banana.ss}.

In the following, we will for convenience use
the following terms:

\item{\bull} {\em input file}: a file to be translated;
\item{\bull} {\em output file}: a file that is the result of
a translation;
\item{\bull} {\em target dialect}: the dialect translated to.

\n In our example above, \p{apple} is an input
file, \p{my-apple} is its corresponding output file,
and Guile is the target dialect.

\subsection{Dialect-configuration files}

The output  file \p{my-apple} above uses Scmxlate's
default rules for an MzScheme-to-Guile translation.
These rules are general and cannot be expected to cover
any peculiar translational information that may be
relevant to the code in \p{apple}.  You can supply such
additional information to Scmxlate via a {\em
dialect-configuration file} called \p{guile-apple} in
the \p{dialects} subdirectory.   Ie, the name of
the dialect-configuration file for a given input file
and a given dialect is formed from  the Scmxlate symbol
for the dialect, followed by a hyphen, followed by the
name of the input file.

Scmxlate typically takes code from a
dialect-configuration file and sticks it  ahead of the
translated code in the output file.  This code can be
any Scheme code in the target dialect, and in
particular, it can include definitions.  The order of
the code in the dialect-configuration file is preserved
in the output file.

For instance, if the MzScheme code in \p{apple} made
use of a nonstandard (MzScheme-only) primitive such as
\q{file-or-directory-modify-seconds}, we could supply
the following Guile definition in the
dialect-configuration file,
\path{dialects/guile-apple}:

\q{
(define file-or-directory-modify-seconds
  (lambda (f) (vector-ref (stat f) 9)))
}

If the dialect-configuration file supplies a definition for
a name that is also defined in the input file,
then the output file will contain the definition from
the dialect-configuration file, not the input file.
For example, if \p{apple} contained
the definition

\q{
(define file-newer?
  (lambda (f1 f2)
    ;checks if f1 is newer than f2
    (> (file-or-directory-modify-seconds f1)
       (file-or-directory-modify-seconds f2))))
}

\n we could put a competing Guile-specific definition
in \p{dialects/guile-apple}:

\q{
(define file-newer?
  (lambda (f1 f2)
    (> (vector-ref (stat f1) 9)
       (vector-ref (stat f2) 9))))
}

\n When Scmxlate translates \p{apple}, it will directly
incorporate this Guile definition into the output file
\p{my-apple} and won't even attempt to translate
the MzScheme definition of the same name in the
input file.

\subsection{Target dialects}

In the above, we used the symbol \q{guile} in the
\p{dialects/dialects-supported.scm} file to signal to
Scmxlate that Guile is one of the dialects into which
the package can be translated.  The list of dialect symbols
recognized by Scmxlate is: \q{bigloo}, \q{chez},
\q{cl},
\q{gambit}, \q{gauche}, \q{guile}, \q{kawa}, \q{mitscheme},
\q{mzscheme}, \q{other}, \q{petite}, \q{pscheme}, \q{scheme48},
\q{scm}, \q{sxm}, \q{scsh}, \q{stk}, \q{stklos},
\q{umbscheme}.

The symbols \q{mzscheme} and \q{plt}
may both be used for PLT Scheme: two symbols are
provided in case two distinct types of translations are
called for --- with \q{mzscheme} perhaps being used to create a
self-sufficient MzScheme script file, and \q{plt} to construct a
PLT module library.

The symbol \q{cl} stands  for
Common Lisp.\f{Note that
Scmxlate can readily determine if it's running
on Common Lisp (as opposed to Scheme), so it will not query the user
for further ``dialect'' information.}

The symbol \q{other} can be used by the package author
to provide a default configuration for an unforeseen
dialect.  Since the dialect is unknown, there isn't
much information to exploit, but it may be
possible to provide some bare-minimum functionality
(or at least display some advice).

The package author can make use of other symbols to
denote other Scheme dialects.  However, as Scmxlate
cannot do any special translation for such dialects, it
is the responsibility of the package author to provide
additional configuration information for them by
writing dialect-configuration files.

\subsection{User-configuration files}

Some packages need some configuration information that
the package author cannot predict and that therefore
can come only come from the user.  The information
typically contains user preferences for global
variables in the program.  It should not be
dialect-specific.

Such user information can be placed in {\em
user-configuration files} in the package directory.
Each input file can have its own
user-configuration file, and the latter's name
consists of the prefix \p{scmxlate-} followed by the
name of the input file.  Thus the user configuration
file for \p{orange.scm} is \p{scmxlate-orange.scm}.

While the package author may not be able to predict the
values of the globals preferred by their various
users, they can include in the package sample
user-configuration files that mention the globals
requiring the user's intervention, with comments
instructing how the user is to customize them.

Note that user-configuration code comes ahead of the
dialect-configuration code in the output file.
Definitions in the user-configuration code override
definitions in the dialect-configuration code, just
as the latter themselves override definitions in the
input file.

\section{The Scmxlate directives}
\label{glossary}

In addition to Scheme code intended to either augment
or override code in the input file, the
dialect- and user-configuration files can
use a small set of Scmxlate directives to finely control
the text that goes into the output file, and even
specify actions that go beyond the mere creation
of the output file.  These directives are now described.

\re{{\tt scmxlate-insert}}

As we saw, Scheme code in the dialect- and
user-configuration files is transferred verbatim
to the output file.  Sometimes, we need to put into the
output file arbitrary text that is not Scheme code.
For instance, we may want the output file to start with
a ``shell magic'' line, so that it can be used as a
shell script.  Such text can be written using the
\p{scmxlate-insert} directive, which evaluates its
subforms in Scheme and displays them on the output
file.

Eg, if you put the following at
the very head of the \p{guile-apple} file:

\q{
(scmxlate-insert
  "#!/bin/sh
exec guile -s $0 \"$@\"
!#
")
}

\n the output Guile file \p{my-apple} will start with the
line

\p{
#!/bin/sh
exec guile -s $0 "$@"
!#
}

Note that the order of the code and \q{scmxlate-insert}
text in the configuration file is preserved in
the output file.

\re{{\tt scmxlate-postamble}}

Typically, the Scheme code and \p{scmxlate-insert}s
specified in the dialect-configuration file occur in
the output file before the translated counterpart of
input file's contents, and thus may be considered as
{\em preamble} text.   Sometimes we need to add {\em
postamble} text, ie, things that go {\em after} the
code from the input file.  In order to do this,
place the directive

\q{
(scmxlate-postamble)
}

\n after any preamble text in the dialect-configuration
file.  Everything following that, whether Scheme
code or \q{scmxlate-insert}s, will show up in the
output file after the translated contents of the input
file.

\re{{\tt scmxlate-postprocess}}

One can also specify actions that need to performed
after the output file has been written.  Eg, let's say
we want the Guile output file for \p{apple} to be
named \p{pear} rather than \p{my-apple}.  We can
enclose Scheme code for achieving this inside the
Scmxlate directive \q{scmxlate-postprocess}:

\q{
(scmxlate-postprocess
  (rename-file "my-apple" "pear"))
}

\re{{\tt scmxlate-ignore-define}}

Sometimes the input file has a definition that the
target dialect does not need, either because the target
dialect already has it as a primitive, or because we
wish to completely re-write input code that uses that
definition.  Eg, if the target dialect is MzScheme,
which already contains \q{reverse!}, any definition of
\q{reverse!} in the input file can be ignored.

\q{
(scmxlate-ignore-define reverse!)
}

\q{scmxlate-ignore-define} can have any number of
arguments.  The definitions of all of them will be
ignored.

\re{{\tt scmxlate-rename}}

Sometimes we want to rename certain identifiers from
the input file.  One possible motivation is that
these identifiers name nonstandard primitives that are
provided under a different name in the target dialect.
For instance, the Bigloo versions of the MzScheme
primitives \q{current-directory} and
\q{file-or-directory-modify-seconds} are \q{chdir} and
\q{file-modification-time} respectively.  So if your
MzScheme input file uses \q{current-directory} and
\q{file-or-directory-modify-seconds}, your Bigloo
dialect-configuration file should contain

\q{
(scmxlate-rename
 (current-directory chdir)
 (file-or-directory-modify-seconds file-modification-time))
}

Note the syntax: \q{scmxlate-rename} has any number of
twosomes as arguments.  The left item is the name in
the input file, and the right item is its proposed
replacement.

\re{{\tt scmxlate-rename-define}}

Sometimes the input file includes a definition
for an operator that the target dialect already has as
a primitive, but with a different name.   Eg, consider
an input file that contains a definition for
\q{nreverse}.  MzScheme has the same operator but with
name \q{reverse!}.  You could add the following to
the MzScheme dialect-configuration file:

\q{
(scmxlate-rename-define
  (nreverse reverse!))
}

Note that this is shorthand for

\q{
(scmxlate-ignore-define nreverse)
(scmxlate-rename
  (nreverse reverse!))
}
\re{{\tt scmxlate-prefix}}

Another motivation for renaming is to avoid polluting
namespace.  We may wish to have short names in the
input file, but when we configure it, we want longer,
``qualified'' names.  It is possible to use
\q{scmxlate-rename} for this, but the
\q{scmxlate-prefix} is convenient when the newer names
are all uniformly formed by adding a prefix.

\q{
(scmxlate-prefix
  "regexp::"
  match
  substitute
  substitute-all)
}

\n renames the identifiers \q{match}, \q{substitute},
and \q{substitute-all} to
\q{regexp::match}, \q{regexp::substitute}, and
\q{regexp::substitute-all} respectively.

The first argument of \q{scmxlate-prefix} is the
string form of the prefix; the remaining arguments are
the identifiers that should be renamed.

\re{{\tt scmxlate-cond}}

Sometimes we want parts of the dialect-configuration
file to processed only when a condition holds.  For
instance, we can use the following \q{cond}-like
conditional in
a dialect-configuration file for MzScheme to
write out a  shell-magic
line appropriate to the operating system:

\q{
(scmxlate-cond
  ((eqv? (system-type) 'unix)
   (scmxlate-insert *unix-shell-magic-line*))
  ((eqv? (system-type) 'windows)
   (scmxlate-insert *windows-shell-magic-line*)))
}

\n where \q{*unix-shell-magic-line*} and
\q{*windows-shell-magic-line*} are replaced by
appropriate strings.

Note that while \q{scmxlate-cond} allows the \q{else}
keyword for its final clause, it does not support the
Scheme \q{cond}'s \q{=>} keyword.

\re{{\tt scmxlate-eval}}

The test argument of \q{scmxlate-cond} and all the
arguments of \q{scmxlate-insert} are evaluated in the
Scheme global environment when Scmxlate is running.
You can enhance this environment with
\q{scmxlate-eval}.  Thus, if we had

\q{
(scmxlate-eval
  (define *unix-shell-magic-line* <...>)
  (define *windows-shell-magic-line* <...>))
}

\n where the \q{<...>} stand for code that constructs
the appropriate string, then we could use these
variables as the arguments to \q{scmxlate-insert} in
the example under \q{scmxlate-cond}.

\q{scmxlate-eval} can have any number of subforms.
It evaluates each of them in the given order.

\re{{\tt scmxlate-compile}}

\q{scmxlate-compile} can be used to tell if the output
file is to be compiled.  Typical usage is

\q{
(scmxlate-compile #t)      ;or
(scmxlate-compile #f)
}

\n The first forces compilation but only if the dialect
supports it, and the second disables compilation even
if the dialect supports it.  The argument of
\q{scmxlate-compile} can be any expression, which is
evaluated only for its boolean significance.

Without a \q{scmxlate-compile} setting, Scmxlate will
ask the user explicitly for advice, but only if
the dialect supports compilation.

\re{{\tt scmxlate-include}}

It is often convenient to keep in a separate file some
of the portions of the text that should go into a
dialect-configuration file.  Some definitions may
naturally be already written down somewhere else, or
we may want the text to be shared across several
dialect-configuration files (for different dialects).
The call

\q{
(scmxlate-include "filename")
}

\n inserts the contents of \q{"filename"}
into that location in the dialect-configuration file.

\re{{\tt scmxlate-uncall}}

It is sometimes necessary to skip a top-level
call when translating an input file.  For instance,
the input file may be used as a script file whose
scriptural action consists in calling a procedure
called \q{main}.  The target dialect may not allow
the output file to be a script, so the user may prefer
to load the output file into Scheme as a library
and make other arrangements to invoke its
functionality.  To disable the call to \q{main}
in the output file, add

\q{
(scmxlate-uncall main)
}

\n to the configuration file.

\q{scmxlate-uncall} can take any number of symbol
arguments.  All the corresponding top-level calls
will be disabled in the output.

\bye

Some rejected crap follows.

Only two of these symbols need special explanation: A
user can pick the \p{other} dialect if his Scheme isn't
listed in the choices that Scmxlate offers.  The
dialect \p{cl} isn't a Scheme dialect but Common Lisp.
Scheme dialects do identified  by human
intervention, as there is (yet) no portable Scheme code
to id the dialect.

More than one file can be configured using
Scmxlate.  Just add the filenames to
\p{dialects/files-to-be-ported.scm}.  Customizing info
tailored to each file can be added to the \p{dialects}
directory as we have already described for the
file \p{progfile}.  Ie, an Scsh customization file
for \p{anotherfile.ss} would be
\p{dialects/scsh-anotherfile.ss}.

\iffalse
This kind of definition replacement is particularly
useful when the target language is Common Lisp.
For instance, let's say \p{progfile} contains
the definition

\q{
(define lassoc
  (lambda (k al equ?)
    (let loop ((al al))
      (if (null? al) #f
          (let ((c (car al)))
            (if (equ? (car c) k) c
                (loop (cdr al))))))))
}

Scmxlate will provide a complicated if working
Common Lisp translation of the above code, but it
will not be as simple as

\q{
(defun lassoc (k al equ?)
  (assoc k al :test equ?))
}

You can put this latter definition in
\p{dialects/cl-progfile} -- where the symbol \q{cl}
stands for Common Lisp -- and it will be used in
preference to the default translation.
\fi

You may wish for some extra CL code to precede or
follow the translated \p{progfile} code.  For instance,
you may wish to add some additional definitions before
the translation to cover some MzScheme-specific
procedures you may have used in \p{progfile}.  Eg,

\q{
(defun getenv (ev)
  (system::getenv ev))
}

\n This can be placed in the file
\p{dialects/cl-preamble-progfile}.

CL code you want following the \p{progfile} code can be
placed in \p{dialects/cl-postamble-progfile}.

\subsection{Specifying Scheme dialects}

The filename prefix \p{cl-} used in the previous
example is used to identify configuration info
for Common Lisp.  If the target language is another
Scheme dialect, rather than Common Lisp, you can follow
a similar procedure, except that we need some way for
the target Scheme to identify itself to Scmxlate.
Scmxlate can tell if it is running in CL, but
needs help in determining which particular Scheme
dialect it is running.

For example, let's say the target Scheme dialect
is Guile.  We create in \p{dialects} a file called
\p{dialects-supported.scm} containing the line

\p{
guile
}

Now if the user starts Guile in the \p{pkgdir}
directory and loads \p{scmxlate.scm}, the following
question will be asked:

\p{
What is your Scheme dialect?
     (guile other)
}

Typing \p{guile} in response will cause Scmxlate to
create a \p{my-progfile} that is the Guile translation
of \p{progfile}.  You can add additional Guile
configuration info in the form of the files
\p{guile-preamble-progfile}, \p{guile-procs-progfile},
and \p{guile-postamble-progfile}, exactly as for CL
above.

The user can use the symbol \p{other} if his Scheme
dialect is not listed in \p{dialects-supported.scm} but
he wants to configure the package anyway.  The results
may be variable.  The configurer can also put in
additional config info in the \p{dialects}
directory using the \p{other-} prefix.

You can certainly add a symbol of your own in
\p{dialects-supported.scm}.  Scmxlate will not know
of it by default, but you can add additional
configuration files using the appropriate prefix in
\p{dialects}.

\subsection{Configuring more than one file}

You can of course configure more than one \p{progfile}.
Simply add their names to the
\p{dialects/files-to-be-ported.scm} directory.  By
default, the translated files will have the same names
as the originals, but with the prefix \p{my-} in front
of them.

\subsection{To be described}

Scmxlate-specific commands used in the
configuration files:

user-override-file

operating-system dependencies

Let us say you wrote a Scheme file named
\p{progfile}\f{The Scheme file's name may have no or
any extension.  Thus, \p{newton-raphson},
\p{newton-raphson~}, \p{newton-raphson.bak},
\p{newton-raphson.scm}, \p{newton-raphson.ss},
\p{newton-raphson.java} are all acceptable filenames
--- but the file's contents must be Scheme code.} in a
directory \p{pkgdir}, and you package it off into a
distribution which an end-user will unpack to
produce a directory \p{pkgdir} of his own.
