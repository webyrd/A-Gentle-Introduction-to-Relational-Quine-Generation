\documentclass[11pt]{book}

\usepackage{alltt}
\usepackage{amsfonts}
\usepackage{amsmath}
\usepackage{amssymb}
\usepackage{ccicons}
\usepackage[cspex,bbgreekl]{mathbbol}
\usepackage{slatex}
\usepackage{stmaryrd}
\usepackage{url}
\usepackage{verbatim}

\begin{document}
\begin{schemeregion}

\title{Relational Programming in miniKanren}
\author{William E. Byrd}
\maketitle
\tableofcontents

\newpage
\huge
\ccLogo
\ccAttribution
%\ccby
This work is licensed under a Creative Commons Attribution 3.0 Unported License.
(CC BY 3.0) \url{http://creativecommons.org/licenses/by/3.0/}

\normalsize

\chapter{Acknowledgements}

\chapter{Preface}

\chapter{Introduction to Relational Programming}

\chapter{Introduction to miniKanren}

\chapter{Translating from Scheme to miniKanren}
\subsection{A-Normal Form}
\subsection{Defunctionalization}
\subsection{Pattern Matching}

\chapter{Exploring the Chomsky Hierarchy}
\section{Regular Expression Matching}
\section{Deterministic Finite Automata}

\chapter{Relational Interpreters}
\section{Relational Scheme Interpreter}
\subsection{Generating Quines}
\section{Relational CESK Machine}

\chapter{Type Inference}
\section{Type Inhabitation}

\chapter{Relational Program Transformations}
\section{Continuation-Passing Style}

\chapter{Implementing miniKanren}
\section{Unification}
\section{An Embedding in Scheme}
\section{A miniKanren Interpreter}
\section{A Meta-circular miniKanren Interpreter}
\section{A Relational miniKanren Interpreter}


\end{schemeregion}
\end{document}
