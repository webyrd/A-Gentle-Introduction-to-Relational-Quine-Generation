\chapter{Preface}\label{sec:preface}

\vspace{-1cm}
\begin{quote}When someone says ``I want a programming language in which I need only say what I wish done,'' give him a lollipop.
\vspace{-3mm}
\begin{flushright}
---Alan J. Perlis
\end{flushright}
\end{quote}

\noindent
I want a programming language in which I need only say what I wish
done.

I am not the only one.   Researchers and industry programmers are
increasingly interested in \emph{declarative programming}, in which
programmers specify \emph{what} to do, rather than \emph{how} to do
it.  Perhaps the most common examples of declarative programming
languages are database query languages, such as SQL, XQuery, and
Datalog.  The equational reasoning provided by the lazy functional
language Haskell can also be seen as a form of declarative
programming.

For decades logic programming has been heralded as the ultimate in
declarative programming, with the promise of writing programs as
mathematical relations that do not distinguish between ``input'' and
``output'' arguments.  This is illustrated by the traditional ``hello
world!''~example from logic programming, list concatenation.  In the
best-known logic programming language---Prolog---the call \mbox{{\tt append([1,2,3],[4,5],X)}}
produces the answer \mbox{{\tt X = [1,2,3,4,5]}}.
More impressively, the call 
\mbox{{\tt append(X,Y,[1,2,3,4,5])}} can produce all pairs of lists
{\tt X} and {\tt Y} such that \mbox{{\tt X} $++$ {\tt Y = [1,2,3,4,5]}}.

Alas, most interesting Prolog programs are \emph{not} written in a
relational style.  For reasons of efficiency and expressivity, Prolog
programmers tend to use impure ``extra-logical'' features of the
language, such as the cut ({\tt !}), used to prune the search tree,
and {\tt assert}/{\tt retract}, used to modify the global database of
facts.  These features make reasoning about logic programs more
difficult, and can even result in unsound behavior.  Even when these
features are used correctly, they inhibit the ability to treat programs
as relations.

In many ways, the current practice of logic programming is reminiscent
of how functional programming was practiced before Haskell.  Lisp and
ML programmers understood the benefits of coding in a pure functional
style; in a pinch, however, they could resort to variable mutation or
other side-effects.  As with Prolog, these impure operators were used
for efficiency and expressivity.  Haskell's laziness required a more
pure approach, however, which in turn led to the adoption of monads
for encapsulating effects.  By committing to a pure functional style,
Haskell advanced the theory and practice of functional programming,
even for strict languages.
%
\emph{Constraints spur creativity.}

For the past decade my colleagues and I have been developing
\emph{miniKanren} (\url{http://www.minikanren.org/}), a logic programming
language specifically designed for programming in a relational style.
%
During that time I have developed relational interpreters, term
reducers, type inferencers, and theorem provers, all of which are
capable of program synthesis, or of otherwise running ``backwards.''

Over the past three years miniKanren has gained a significant
following in the Clojure community in the form of the popular {\tt
  core.logic} library, which started as a Clojure port of the code in
my dissertation.  {\tt core.logic} is now being used in both industry
and academia.  miniKanren is also growing in popularity in the Racket
community, in the form of the {\tt cKanren} library. miniKanren has
been ported from Scheme to many other languages, including Python,
Ruby, JavaScript, Haskell, and Scala.
