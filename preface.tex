\chapter{Preface}\label{sec:preface}

\vspace{-1cm}

\begin{quotation}
\noindent
\emph{I think the weakest way to solve a problem is just to solve it; that's what they teach in elementary school. In some math and science courses they often teach you it's better to change the problem. I think it's much better to change the context in which the problem is being stated. Some years ago, Marvin Minsky said, ``You don't understand something until you understand it more than one way.'' I think that what we're going to have to learn is the notion that we have to have multiple points of view.}
\begin{flushright}
\vspace{-0.5em}
---Alan C. Kay\footnote{From Kay's Stanford Computer Forum talk, \emph{Predicting the Future}~\cite{Kay:1989}.\\Also at \url{http://www.ecotopia.com/webpress/futures.htm}.}
\end{flushright}
\end{quotation}


% Predicting The Future, Stanford Engineering, Vol. 1, No. 1, Autumn 1989

Alan J. Perlis said, ``A language that doesn't affect the way you think about programming, is not worth knowing.''\footnote{Epigram 19 from ``Epigrams on Programming''~\cite{Perlis:1982:SFE:947955.1083808}.\\Also at \url{http://www.cs.yale.edu/quotes.html}.}
%
The point of view we shall adopt in this book is, ``A program that doesn't run backwards is not worth writing.''

% http://www.cs.yale.edu/quotes.html
%
%  19. A language that doesn't affect the way you think about programming, is not worth knowing. 


% This book is about changing the context in which problems about computation are stated.

% This book is about solving problems, and understanding computation, in the context of relational programming.

