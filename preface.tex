\chapter{Preface}\label{sec:preface}

\vspace{-1cm}

\begin{quotation}
\noindent
\emph{I think the weakest way to solve a problem is just to solve it; that's what they teach in elementary school. In some math and science courses they often teach you it's better to change the problem. I think it's much better to change the context in which the problem is being stated. Some years ago, Marvin Minsky said, ``You don't understand something until you understand it more than one way.'' I think that what we're going to have to learn is the notion that we have to have multiple points of view.}
\begin{flushright}
\vspace{-0.3em}
---Alan C. Kay\footnote{From Kay's 20th anniversary Stanford Computer Forum talk in 1988, ``Predicting the Future''~\cite{Kay:1989}. Also at \url{http://www.ecotopia.com/webpress/futures.htm}.}
\end{flushright}
\end{quotation}


This book is about changing the context in which computational
problems are stated, attempting to understand these problems from the
viewpoint of relational programming.


In elementary school I learned to multiply small whole numbers by
memorization: I memorized the fact that 3 times 4 is 12.
%
I also learned how to handle larger numbers by repeatedly
multiplifying pairs of digits---one digit from each number---so that I
could apply my mental collection of facts.
%
I solved countless multiplication problems without the faintest idea
that I was using ideas fundamental to computing: following an
algorithm, lookuping up pre-computed partial results, using recursion
to simplify a complex problem.
%
\emph{I learned to just solve the problem.}


Years later I learned that numbers can be represented in binary: the
multiplication problem \mbox{$3 \times 4$} can be changed into the
equivalent binary multiplication problem \mbox{$11_2 \times 100_2$}.
%
I also learned that multiplying a number by 4 in binary is equivalent
to adding two 0's to the end of that number: $11_2$ multiplied by 4 is
$1100_2$, which is the binary representation of 12.
%
On many computers, using a ``shift left'' instruction to add two 0's to
the end of a binary number is faster than multiplying the number by 4.
%
\emph{I learned it is often better to change the problem.}






% What happens if we take the relational view of computation seriously?

% changing context: what is so special about the 3 and 4?  Why do they get all the fun?  Why is 12 left out in the cold?  Isn't 12 just as important as 3 and 4?

Even later, in college, I learned that multiplication can be
described in the context of mathematical relations: the relation
defining multiplication is a collection of ``triples'' of numbers,
such as \mbox{$(3\; 4\; 12)$}.
%
These triples correspond, of course, to the multiplication facts I
learned in elementary school, with one key difference: the
multiplication relation includes \emph{all} of the infinitely many
triples \mbox{$(X\; Y\; Z)$} for which \mbox{$X \times Y = Z$}.
%
From this perspective, the multiplication problem \mbox{$3 \times 4$}
is represented as the triple \mbox{$(3\; 4\; Z)$}, where $Z$ is a
variable with an unknown value.
%
Multiplying 3 by 4 is equivalent to finding a triple in the
multiplication relation that matches \mbox{$(3\; 4\; Z)$}---in this
case, the triple \mbox{$(3\; 4\; 12)$} matches, telling us that the
solution is 12.

The relational view of multiplication is fascinating because we place
variables in the first and second positions of a triple.
%
For example, the triple \mbox{$(X\; 4\; 12)$} represents both the
multiplication problem \mbox{$X \times 4 = 12$} and the division
problem \mbox{$12 \div 4 = X$}.
%
Since this triple matches \mbox{$(3\; 4\; 12)$}, the solution to both
problems is $X = 3$.
%
The triple \mbox{$(X\; Y\; 12)$} matches multiple elements of the
multiplication relation, including \mbox{$(1\; 12\; 12)$},
\mbox{$(3\; 4\; 12)$}, and \mbox{$(6\; 2\; 12)$}.
%
And, of course, the triple \mbox{$(X\; Y\; Z)$} matches each of the
infinitely many triples in the mulplication relation.


In graduate school I learned that this relational view of
multiplication has a computational interpretation.
%
If we restrict the multiple relation to include only a finite number
of triples, by setting an upper bound on the size of the numbers in
each triple, then finding all triples that match \mbox{$(X\; Y\; 12)$}
is equivalent to performing a database query.




\emph{I learned it is much better to change the context in which the problem is being stated.}





Alan J. Perlis said, ``A language that doesn't affect the way you think about programming, is not worth knowing.''\footnote{Epigram 19 from ``Epigrams on Programming''~\cite{Perlis:1982:SFE:947955.1083808}.\\Also at \url{http://www.cs.yale.edu/quotes.html}.}
%
The point of view we shall adopt in this book---the context in which we will consider every problem---is:

\large
{\bf A program that isn't relational is not worth writing.}
\normalsize
