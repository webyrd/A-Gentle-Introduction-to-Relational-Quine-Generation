\chapter{Preface}

\section{Audience}\label{sec:preface:audience}

This\marginnote{}  book is written for
intermediate-to-advanced programmers, computer science students, and
researchers.  For this book, \emph{intermediate} means that you are
comfortable writing simple recursive procedures in a functional
programming language, such as Scheme, Racket, Clojure, Lisp, ML, or
Haskell.  I also assume you have a reading knowledge of Scheme.
%
No knowledge of relational programming, logic programming, programming
language theory, or miniKanren is required.
%
You must, however, prepare to have your mind blown by
the awesomeness that is relational programming.

If you want to learn about relational programming, but are new to
programming, are you in luck!  I co-wrote another book just for you:
\emph{The Reasoned Schemer}\cite{trs}. You might first read \emph{The Little
  Schemer}\cite{Friedman:1996:LS:230223}, a very gentle introduction to recursion and
functional programming.

If you are an experienced programmer, but weak on recursion, you, too, might
benefit from \emph{The Little Schemer}.  If you are comfortable with
recursion, but not functional programming, good introductions include
\emph{Scheme and the Art of Programming}\cite{Springer:1989:Art} and
the classic \emph{Structure and Interpretation of Computer
  Programs}\cite{Abelson:1996:SIC:547755}\marginnote{(full text at
\url{http://mitpress.mit.edu/sicp/full-text/book/book.html})}.

If you are an experienced functional programmer, but do not know
Scheme, the beginning of \emph{Structure and Interpretation of
  Computer Programs} should get you up to speed, while \emph{The
  Scheme Programming Language, 4th Edition}\cite{Dybvig:2009:SPL:1618542}\marginnote{(full text at
\url{http://www.scheme.com/tspl4/})} describes the language in detail.


\section{Margin Notes}\label{sec:preface:margin-notes}

This book is typeset using the ``Tufte-Style Book''
\LaTeX\ style\marginnote{Tufte-Style Book is freely available from
  \url{http://www.LaTeXTemplates.com}.}, based on Edward Tufte's
magnificent \emph{The Visual Display of Quantitative
  Information}\cite{Tufte:1986:VDQ:33404}.  A distinguishing feature
of Tufte's writing is his heavy use of margin notes.  Like Tufte I
love margin notes, which is why I chose this
\LaTeX\ style.\marginnote{True marginalia afficianados should grab a
  continuation, then immediately read David Foster Wallace's ``Host'',
  in:\cite{dfw:lobster:2005}}
%
Margin notes also help solve the problem of addressing readers with
widely varying knowledge of computer science and programming.
%
To make the book accessible to a wide audience, in the main text I
assume the reader is the hypothetical \emph{intermediate-level} programmer or
student described in the \emph{Audience} section above.
%
In the margin notes, however, anything goes.
%
Many margin notes will be aimed at all readers, but some will assume
specialized knowledge of theoretical computer science, programming
languages, or programming idioms.
%
If you find a margin note assumes knowledge you do not have, please
ignore the note and read on!






%   Targeted level of theoretical CS knowledge for existing miniKanren resources
% <----------------------------------------------------------------------------->
% non-existent   basic            intermediate     Undergrad     Graduate     PhD
%                programming      hobbyist         CS            CS           in
%                literacy                          degree        degree       PL
%
%
%              |----------- The Reasoned Schemer -----------|
%
%                                             |---- Weekly Google Hangouts ----|
%
%                                                 |--- My PhD dissertation ----|
%
%                                                 |----- Clojure/conj,
%                                                        Clojure/West,
%                                                        Strange Loop, 
%                                                        Flatmap talks --------|
%
%                                                      |--- Academic Papers ---|
%
%
%                                      |------------ *this book* --------------|
%                                      |--- main text -----|-- margin notes ---|

% Rules: try to target intermediate-to-advanced hobbyists, working
% programmers, and undergrad students in main text.  Allowed to geek
% out in margin notes: anything can go!  Intro chapters and appendices
% will help bring intermediate reader up to speed on CS (and,
% especially, PL-related) topics, such as interpreters, continuations,
% type theory, and program transformations.  This shouldn't be too
% boring, though, since can followup these chapters with the same
% ideas demonstrated relationally (for example, relational
% interpreters, relational CSPer, etc.)  Advanced, optional chapters
% and appendices can explore topics of arbitrary complexity.



%%% SPJ style: describe contributions first.  Then, write to satisfy
%%% the contributions.  I'm done writing once I've backed up all of
%%% the claimed contributions.


% An important goal of the book is to make other resources on
% miniKanren more accessible.  PL-specific terms should be described,
% or at least the reader should be given very specific pointers to
% additional, accessible reading.


%%% I should make sure I have plenty of examples of relational
%%% programming not drawn from PL theory, to make the book more
%%% accessible.  Automata and regex are good examples.  Need to look
%%% for more.  Can borrow some ideas from Prolog books.


% Critical questions that should be answered in the preface:

% What is the goal of the book?  What do I want the reader to get
% out of the book, think about, or be able to do after reading?

% Who is the intended audience?  (Always keep this in mind...)  For
% the Little books, the intended reader is a bright but ignorant high
% school student.

% What is the required background of the reader?  Tied closely to the
% first question, obviously.





% quote?

% what is the origin of this book?
% Talk about TRS, thesis, papers, talks, C311/B521

% Goals for the book/what the reader should be able to do after reading the book

% point of view of the book

% language choice/host language/variants of miniKanren

% special instructions to readers?
% what should the reader know?
% typographic conventions?
% how does this book fit in with TRS and my dissertation?
% what parts of this work have previously appeared, and where?

% where can the reader find the code (and the latest version of the
% book)?  How would the reader run the code?

% structure of the book


% Acknowledgements

%% mK developers/co-authors from previous works

%% C311/B521 students

%% IU grad students

%% Google Hangout participants

%% Clojure community

%% creators of Scheme

%% pioneers in logic programming


%% Structure of TAPL preface:
% One para super-brief description of topic of book.
% \section{Audience}
% \section{Goals}
% \section{Structure} % (with chapter dependency figure)
% \section{Required Background}
% \section{Course Outlines}
% \section{Exercises}
% \section{Typographic Conventions}
% \section{Electronic Resources}
% \section{Acknowledgements}
% Quotes at the end

\wspace

\noindent
William E. Byrd\\
\noindent
Salt Lake City, Utah\\
\noindent
\monthyear
