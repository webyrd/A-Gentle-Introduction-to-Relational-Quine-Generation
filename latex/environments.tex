\chapter{Environments}

An \emph{environment} maps variables to values.
%
We will use an association list representation, with variables
represented as symbols.\footnote{Other representations of environments
  are certainly possible.  For example, we could use separate flat
  lists of corresponding variables and values.  We could use more
  sophisticated representations in Scheme, such as tree-based data
  structures for faster lookup, but this would complicate the
  miniKanren implementation significantly.}
For example, the environment containing the bindings
%
$x \rightarrow$ \schemeresult|(foo)|,
%
$y \rightarrow$ \schemeresult|6|,
%
and 
%
$z \rightarrow$ \schemeresult|#t|
%
might be represented as any of the following association lists (among other possibilities):

\noindent\mbox{\schemeresult|((x . (foo)) (z . #t) (y . 6))|}\footnote{This association list would be displayed in most Scheme systems as \mbox{\schemeresult|((x foo) (z . #t) (y . 6))|}.},

\noindent\mbox{\schemeresult|((y . 6) (x . (foo)) (z . #t))|}

\noindent\mbox{\schemeresult|((y . 6) (z . #t) (x . (foo)) (z . 11))|}

\noindent
In the last association list $z$ is bound to \schemeresult|#t|, since the pair \mbox{\schemeresult|(z . #t)|}
occurs in the list before the \emph{shadowed} association \mbox{\schemeresult|(z . 11)|}.

To look up the value bound to a variable, we will use the \scheme|lookup| function:

\noindent\scheme{(lookup 'z '((z . 5)))} $\Rightarrow$
\begin{schemeresponsebox}5\end{schemeresponsebox}

\noindent\scheme{(lookup 'z '((z . 5) (z . 6)))} $\Rightarrow$
\begin{schemeresponsebox}5\end{schemeresponsebox}

\noindent\scheme{(lookup 'z '((w . 7) (z . 5) (z . 6)))} $\Rightarrow$
\begin{schemeresponsebox}5\end{schemeresponsebox}

A straight-forward definition of \scheme|lookup| in Scheme using
pattern matching might look like this:

\begin{schemedisplay}
(define lookup
  (lambda (x env)
    (pmatch env
      (() (error 'lookup "unbound variable"))
      (((,y . ,v) . ,rest)
       (cond
         ((eq? y x) v)
         (else (lookup x rest)))))))
\end{schemedisplay}

However, this version will cause problems if we were to translate it
to miniKanren.  We can see this by noting that the \scheme|cond|
clauses cannot be reordered.  We could fix this by replacing the
\scheme|else| with the negation of the \mbox{\scheme|(eq? y x)|} test
in the previous \scheme|cond| clause.

\begin{schemedisplay}
(define lookup
  (lambda (x env)
    (pmatch env
      (() (error 'lookup "unbound variable"))
      (((,y . ,v) . ,rest)
       (cond
         ((eq? y x) v)
         ((not (eq? y x)) (lookup x rest)))))))
\end{schemedisplay}

Then it would be safe to reorder the \scheme|cond| clauses.  Of course, we can
reorder the \scheme|pmatch| clauses as well:

\begin{schemedisplay}
(define lookup
  (lambda (x env)
    (pmatch env
      (((,y . ,v) . ,rest)
       (cond
         ((not (eq? y x)) (lookup x rest))
         ((eq? y x) v)))
      (() (error 'lookup "unbound variable")))))
\end{schemedisplay}

\noindent\scheme{(lookup 'z '((w . 7) (z . 5) (z . 6)))} $\Rightarrow$
\begin{schemeresponsebox}5\end{schemeresponsebox}

need to make sure each version of lookup is properly tested in the src directory;
should probably break the environment helpers into a separate source code file


better definition: check that \scheme|x| and \scheme|y| are indeed
symbols, and can reorder clauses.  Also arguably cleaner
stylistically, since is doesn't mix \scheme|pmatch| and \scheme|cond|
unnecessarily.  However, there is some duplication in the patterns of
the last two clauses.

\begin{schemedisplay}
(define lookup
  (lambda (x env)
    (unless (symbol? x)
      (error 'lookup "first argument must be a symbol"))
    (pmatch env
      (() (error 'lookup "unbound variable"))
      (((,y . ,v) . ,rest) (guard (symbol? y) (eq? y x))
       v)
      (((,y . ,v) . ,rest) (guard (symbol? y) (not (eq? y x)))
       (lookup x rest)))))
\end{schemedisplay}

Tempting to instead write:

\begin{schemedisplay}
(define lookup
  (lambda (x env)
    (unless (symbol? x)
      (error 'lookup "first argument must be a symbol"))
    (pmatch env
      (() (error 'lookup "unbound variable"))
      (((,x . ,v) . ,rest)
       v)
      (((,y . ,v) . ,rest) (guard (symbol? y) (not (eq? y x)))
       (lookup x rest)))))
\end{schemedisplay}

However, this definition does \emph{not} work, since the \scheme|x| in
the pattern of the second clause shadows the first argument to
\scheme|lookup|

(need more explanation of this; also, this shadowing is at a different
level than the shadowing in an association list)




\begin{schemedisplay}
(define lookupo
  (lambda (x env val)
    (fresh ()
      (symbolo x)
      (matche env
        (((,y . ,v) . ,rest) (symbolo y)
         (== y x) (== v val))
        (((,y . ,v) . ,rest) (symbolo y)
         (=/= y x) (lookupo x rest val))))))
\end{schemedisplay}

\noindent\scheme{(run* (q) (lookupo 'z '((z . 5) (z . 6)) q))} $\Rightarrow$
\begin{schemeresponsebox}(5)\end{schemeresponsebox}

\noindent\scheme{(run* (q) (lookupo 'z '((w . 7) (z . 5) (z . 6)) q))} $\Rightarrow$
\begin{schemeresponsebox}(5)\end{schemeresponsebox}

\noindent\scheme{(run* (q) (lookupo 'y '((x . foo) (y . bar)) q))} $\Rightarrow$
\begin{schemeresponsebox}(bar)\end{schemeresponsebox}

\noindent\scheme{(run* (q) (lookupo 'w '((x . foo) (y . bar)) q))} $\Rightarrow$
\begin{schemeresponsebox}()\end{schemeresponsebox}

\noindent\scheme{(run 5 (q) (lookupo 'z q 5))}

$\Rightarrow$

\begin{schemeresponsebox}
(((z . 5) . _.0)
 (((_.0 . _.1) (z . 5) . _.2)
  (=/= ((_.0 z))) (sym _.0))
 (((_.0 . _.1) (_.2 . _.3) (z . 5) . _.4)
  (=/= ((_.0 z)) ((_.2 z))) (sym _.0 _.2))
 (((_.0 . _.1) (_.2 . _.3) (_.4 . _.5) (z . 5) . _.6)
  (=/= ((_.0 z)) ((_.2 z)) ((_.4 z))) (sym _.0 _.2 _.4))
 (((_.0 . _.1) (_.2 . _.3) (_.4 . _.5) (_.6 . _.7) (z . 5) . _.8)
  (=/= ((_.0 z)) ((_.2 z)) ((_.4 z)) ((_.6 z))) (sym _.0 _.2 _.4 _.6)))
\end{schemeresponsebox}

These \scheme|run| expressions return the same answers if we reorder the \scheme|matche| clauses in \scheme|lookupo|:

\begin{schemedisplay}
(define lookupo
  (lambda (x env val)
    (fresh ()
      (symbolo x)
      (matche env
        (((,y . ,v) . ,rest) (symbolo y)
         (=/= y x) (lookupo x rest val))
        (((,y . ,v) . ,rest) (symbolo y)
         (== y x) (== v val))))))
\end{schemedisplay}




\setbox\boxa\vtop{{\footnotesize
\begin{schemebox}
(defmatche (lookupo x env val)
  ((,x ((,x . ,val) . ,rest) ,val)
   (symbolo x))
  ((,x ((,y . ,v) . ,rest) ,val)
   (symbolo x) (symbolo y)
   (=/= x y) (lookupo x rest val)))
\end{schemebox}}}

\footnote{Alternatively, we could define \scheme|lookupo| using \scheme|defmatche|, which arguably results in a cleaner definition: \\ \ \\ \usebox{\boxa} \\ \ \\ However, this definition does not as clearly reflect the structure of the \scheme|lookup| function defined using \scheme|pmatch|.}
