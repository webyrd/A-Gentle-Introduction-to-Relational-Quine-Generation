%\newcommand{\bookauthor}[0]{William E. Byrd}
%\newcommand{\booktitle}[0]{Relational Programming in miniKanren}
\newcommand{\bookkeywords}[0]{miniKanren, Relational Programming, Logic Programming}

\newcommand{\wspace}{{\vspace{1em}}}

\usepackage{morefloats}

\usepackage{alltt}
\usepackage{amsfonts}
\usepackage{amsmath}
\usepackage{amssymb}

% For the XeLaTeX symbol
\usepackage{xltxtra}

% Creative Commons license icons
\usepackage{ccicons}

\usepackage{comment}

\usepackage{float}

% Load Palatino-like fonts
% from http://tex.stackexchange.com/questions/50586/how-to-use-palatino-like-math-fonts-in-xelatex
\usepackage{fontspec}
\defaultfontfeatures{Ligatures=TeX} % makes this a feature for all selected fonts
\usepackage{mathpazo}
\setmainfont
     [ BoldFont       = texgyrepagella-bold.otf ,
       ItalicFont     = texgyrepagella-italic.otf ,
       BoldItalicFont = texgyrepagella-bolditalic.otf ]
     {texgyrepagella-regular.otf}

\usepackage{hanging}

% Careful: mathbbol apparently doesn't like 11pt
% The stmryrd error about 11pt is actually caused by mathbbol
% See http://www.latex-community.org/forum/viewtopic.php?f=4&t=2099
% 10pt or 12pt are apparently okay, though.
%\usepackage[cspex,bbgreekl]{mathbbol}

\usepackage{natbib}

\usepackage{slatex}
\usepackage{stmaryrd}
\usepackage{textcomp}

% Uncomment to use author-date citation format
%\bibpunct();A{},
%\let\cite=\citep

\newcommand{\RfiveRS}{$R^{5}\!RS$}
\newcommand{\RfiveRSsp}{$R^{5}\!RS$ }

\newcommand{\RsixRS}{$R^{6}\!RS$}
\newcommand{\RsixRSsp}{$R^{6}\!RS$ }

\newcommand{\lambdaProlog}{{$\lambda$Prolog}}

\newcommand{\Godel}{{G{\"o}del}}



%--------------------------------------------

%%% SLaTeX stuff

% Fake using the Computer Modern font within SLaTeX.
\newcommand{\cmr}{\fontfamily{lmr}\selectfont}

\def\keywordfont#1{{\cmr{\textbf{#1}}}}
\def\variablefont#1{{\cmr{\textit{#1}}}}
\def\constantfont#1{{\cmr{\textsf{#1}}}}
\let\datafont\constantfont

\defschememathescape{$} % $
\def\schemecodehook{\setlength{\baselineskip}{0.5\baselineskip}}

% Scheme/mk typesetting
\setspecialsymbol{lambda}{{\cmr{$\lambda$}}}
\setspecialsymbol{=>}{$\Rightarrow$}

\def\plusosymbol{$+^{o}$}
\setspecialsymbol{+o}{{\plusosymbol}}

\def\timesosymbol{$*^{o}$}
\setspecialsymbol{*o}{{\timesosymbol}}

\def\divosymbol{$/^{o}$}
\setspecialsymbol{/o}{{\divosymbol}}


\def\evalosymbol{{eval$^{\thinspace o}$}}
\setspecialsymbol{evalo}{{\variablefont{\evalosymbol}}}
