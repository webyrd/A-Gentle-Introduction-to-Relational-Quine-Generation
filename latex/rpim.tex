%%%%%%%%%%%%%%%%%%%%%%%%%%%%%%%%%%%%%%%%%
% This file is a modified version of the
%
% Tufte-Style Book (Minimal Template)
% LaTeX Template
% Version 1.0 (5/1/13)
%
% downloaded from: http://www.LaTeXTemplates.com
%
% The license for *this LaTeX file only* (and the original template) is:
% CC BY-NC-SA 3.0 (http://creativecommons.org/licenses/by-nc-sa/3.0/)
% 
% All files for this book, other than this file, and the
% 'tufte-book.cls' and 'tufte-common.cls' files, are covered under the
% license: 
% CC BY 4.0 (http://creativecommons.org/licenses/by/4.0/)
%%%%%%%%%%%%%%%%%%%%%%%%%%%%%%%%%%%%%%%%%

% Letterspacing is apparently busted in the latest Tufte template.
% If you see the error message
%
% ! Argument of \MakeTextUppercase has an extra }.
%
% you can disable letterspacing by adding [nols]:
%
\documentclass[nols]{tufte-book}
%\documentclass{tufte-book} % Use the tufte-book class which in turn uses the tufte-common class

\hypersetup{colorlinks} % Comment this line if you don't wish to have colored links

%\usepackage{microtype} % Improves character and word spacing

\usepackage{booktabs} % Better horizontal rules in tables

\usepackage{graphicx} % Needed to insert images into the document
\graphicspath{{graphics/}} % Sets the default location of pictures
\setkeys{Gin}{width=\linewidth,totalheight=\textheight,keepaspectratio} % Improves figure scaling

\usepackage{fancyvrb} % Allows customization of verbatim environments
\fvset{fontsize=\normalsize} % The font size of all verbatim text can be changed here

\newcommand{\hangp}[1]{\makebox[0pt][r]{(}#1\makebox[0pt][l]{)}} % New command to create parentheses around text in tables which take up no horizontal space - this improves column spacing
\newcommand{\hangstar}{\makebox[0pt][l]{*}} % New command to create asterisks in tables which take up no horizontal space - this improves column spacing

\usepackage{xspace} % Used for printing a trailing space better than using a tilde (~) using the \xspace command

\newcommand{\monthyear}{\ifcase\month\or January\or February\or March\or April\or May\or June\or July\or August\or September\or October\or November\or December\fi\space\number\year} % A command to print the current month and year

\newcommand{\openepigraph}[2]{ % This block sets up a command for printing an epigraph with 2 arguments - the quote and the author
\begin{fullwidth}
\sffamily\large
\begin{doublespace}
\noindent\allcaps{#1}\\ % The quote
\noindent\allcaps{#2} % The author
\end{doublespace}
\end{fullwidth}
}

\newcommand{\blankpage}{\newpage\hbox{}\thispagestyle{empty}\newpage} % Command to insert a blank page

\usepackage{makeidx} % Used to generate the index
\makeindex % Generate the index which is printed at the end of the document

\newcommand{\bookauthor}[0]{William E. Byrd}
\newcommand{\booktitle}[0]{Relational Programming in miniKanren}
\newcommand{\bookkeywords}[0]{miniKanren, Relational Programming, Logic Programming}

\newcommand{\wspace}{{\vspace{1em}}}

\usepackage{alltt}
\usepackage{amsfonts}
\usepackage{amsmath}
\usepackage{amssymb}

% For the XeLaTeX symbol
\usepackage{xltxtra}

% Creative Commons license icons
\usepackage{ccicons}

\usepackage{comment}

\usepackage{float}

% Load Palatino-like fonts
% from http://tex.stackexchange.com/questions/50586/how-to-use-palatino-like-math-fonts-in-xelatex
\usepackage{fontspec}
\defaultfontfeatures{Ligatures=TeX} % makes this a feature for all selected fonts
\usepackage{mathpazo}
\setmainfont
     [ BoldFont       = texgyrepagella-bold.otf ,
       ItalicFont     = texgyrepagella-italic.otf ,
       BoldItalicFont = texgyrepagella-bolditalic.otf ]
     {texgyrepagella-regular.otf}

\usepackage{hanging}

% Careful: mathbbol apparently doesn't like 11pt
% The stmryrd error about 11pt is actually caused by mathbbol
% See http://www.latex-community.org/forum/viewtopic.php?f=4&t=2099
% 10pt or 12pt are apparently okay, though.
%\usepackage[cspex,bbgreekl]{mathbbol}

%\usepackage{natbib}

\usepackage{slatex}
\usepackage{stmaryrd}
\usepackage{textcomp}

% Uncomment to use author-date citation format
%\bibpunct();A{},
%\let\cite=\citep

%--------------------------------------------

%%% SLaTeX stuff

% Fake using the Computer Modern font within SLaTeX.
\newcommand{\cmr}{\fontfamily{lmr}\selectfont}

\def\keywordfont#1{{\cmr{\textbf{#1}}}}
\def\variablefont#1{{\cmr{\textit{#1}}}}
\def\constantfont#1{{\cmr{\textsf{#1}}}}
\let\datafont\constantfont

\defschememathescape{$} % $
\def\schemecodehook{\setlength{\baselineskip}{0.5\baselineskip}}

\setspecialsymbol{lambda}{{\cmr{$\lambda$}}}


%----------------------------------------------------------------------------------------
%	BOOK META-INFORMATION
%----------------------------------------------------------------------------------------

\title{Relational Programming in miniKanren}

\author{William E. Byrd}

%\publisher{Publisher Name}


\begin{document}
\begin{schemeregion}

\frontmatter
%\pagenumbering{roman} % Tufte-Style Book uses Arabic numbering for front matter


%%  Overridden!!
% Produces a full title page

\renewcommand{\maketitlepage}[0]{%
  \cleardoublepage%
  {%
  \rmfamily%
  \begin{fullwidth}%
%  \fontsize{30}{34}\selectfont\par\noindent\textcolor{darkgray}{\thanklesstitle}%
  \fontsize{36}{38}\selectfont\par\noindent\textcolor{darkgray}{Relational Programming\\\noindent in miniKanren}%    
  \vspace{3pc}%
  \sffamily%
  \fontsize{20}{22}\selectfont\par\noindent\textcolor{darkgray}{\allcaps{\thanklessauthor}}%   
  \vfill%  
  \fontsize{14}{16}\selectfont\par\noindent\allcaps{\thanklesspublisher}%
  \end{fullwidth}%
  }

  \thispagestyle{empty}%
  \clearpage%
}

\maketitle % Print the title page

\blankpage

\blankpage

\blankpage

\blankpage

\blankpage

% internal title page
\thispagestyle{empty}

\begin{fullwidth}
\rmfamily  
\itshape
\noindent\fontsize{20}{24}\selectfont
\thanklessauthor

\vspace{10.5pc}%

\upshape
\noindent\fontsize{36}{38}\selectfont
\nohyphenation
%\thanklesstitle
Relational Programming\\\noindent in miniKanren

\normalsize

~\vfill

\end{fullwidth}


%----------------------------------------------------------------------------------------
%	COPYRIGHT PAGE
%----------------------------------------------------------------------------------------

\newpage

\begin{fullwidth}
~\vfill
\thispagestyle{empty}
\setlength{\parindent}{0pt}
\setlength{\parskip}{\baselineskip}
\large
\noindent
\textcopyright~2014 William E. Byrd

\noindent
Typeset by the author in \XeLaTeX, using Tufte-Style Book (\url{http://www.LaTeXTemplates.com}).

\today\xspace version. \\ The latest version of this book, along with all files necessary to typeset it, can be found at: \\ \url{https://github.com/webyrd/relational-programming-in-miniKanren}

\noindent
\huge
\ccLogo
\ccAttribution
\large

\noindent
This work is licensed under a Creative Commons Attribution 4.0 International License.
(CC BY 4.0) \\
\url{http://creativecommons.org/licenses/by/4.0/}
\normalsize
\end{fullwidth}


\tableofcontents


%----------------------------------------------------------------------------------------
%	DEDICATION PAGE
%----------------------------------------------------------------------------------------

\cleardoublepage
\thispagestyle{empty}
\begin{fullwidth}
~\vfill
\begin{doublespace}
\noindent\fontsize{18}{22}\selectfont\itshape
\nohyphenation
\begin{tabular}{ll}
For my H211 students: \\ Indiana University, Fall 2010 \& 2011, and Team pw0ni3.
\end{tabular}
\end{doublespace}
\begin{flushright}
\Large
\textit{Learning with always trumps learning from.}

\wspace

---Woodie Flowers
\normalsize
\end{flushright}
\vfill
\vfill
\end{fullwidth}


\blankpage


\chapter*{Preface}


% include acks here

\vspace{5mm}

William E. Byrd\\
Salt Lake City, Utah\\
\monthyear


\mainmatter
%\pagenumbering{arabic}

\chapter{Introduction}

This book describes the miniKanren programming language\marginnote{miniKanren is actually a family of related programming languages, embedded in a variety of host languages. Unless otherwise specifified, we use the term ``miniKanren'' to refer to the entire family of languages, including {\tt core.logic}, cKanren, and any other variants.}, and how to use miniKanren to write programs in a {\em relational} style.


%\chapter{The miniKanren Language}

%% \vspace{-1.5cm}
%% \begin{fullwidth}
%% \begin{flushright}
%% \Large
%% \textit{gl hf!}

%% \wspace

%% ---Greg ``IdrA'' Fields
%% %(Traditional greeting in the Koprulu Sector)
%% \normalsize
%% \end{flushright}
%% \end{fullwidth}



%\chapter{Finite Automata}

%% \vspace{-1.5cm}
%% \begin{fullwidth}
%% \begin{flushright}
%% \Large
%% \textit{gl hf!}

%% \wspace

%% ---Greg ``IdrA'' Fields
%% %(Traditional greeting in the Koprulu Sector)
%% \normalsize
%% \end{flushright}
%% \end{fullwidth}



\chapter{Relational Interpreters}

\vspace{-1.5cm}
\begin{fullwidth}
\begin{flushright}
\Large
\textit{We will never run out of things to program as long as there is a single program around.}

\wspace

---Alan J. Perlis

\noindent
Epigrams on Programming, \#100\nocite{Perlis:1982:SFE:947955.1083808}
\normalsize
\end{flushright}
\end{fullwidth}


\chapter{Conclusion}

\begin{fullwidth}
\begin{flushright}
\Large
\textit{G.G.}

\wspace

---Sean ``Day[9]'' Plott
\normalsize
\end{flushright}
\end{fullwidth}


%----------------------------------------------------------------------------------------

% description of appendix vs. backmatter ordering:
% http://tex.stackexchange.com/questions/20538/what-is-the-right-order-when-using-frontmatter-tableofcontents-mainmatter

%\appendix

\backmatter

%----------------------------------------------------------------------------------------
%	BIBLIOGRAPHY
%----------------------------------------------------------------------------------------

\bibliography{rpim}
\bibliographystyle{abbrvnat}

%----------------------------------------------------------------------------------------

\printindex % Print the index at the very end of the document

\end{schemeregion}
\end{document}
